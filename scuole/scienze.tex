Scienze Matematiche, Fisiche e Naturali

Presentazione
Scienze ha più di 5000 iscritti e offre 10 corsi di laurea triennale e 11 corsi di laurea magistrale.

Corsi di laurea triennale
     • Biotecnologie (225 posti) 
     • Fisica 
     • Informatica (300)
     • Matematica 
     • Ottica e optometria (150) 
     • Scienza dei materiali 
     • Scienze biologiche (225 posti) 
     • Scienze e tecnologie chimiche (100) 
     • Scienze e tecnologie geologiche 
     • Scienze e tecnologie per l'ambiente (150)

Corsi di laurea magistrale
     • Astrofisica e fisica dello spazio 
     • Biotecnologie industriali 
     • Biologia 
     • Fisica 
     • Informatica 
     • Matematica 
     • Scienza dei materiali 
     • Scienze e tecnologie chimiche 
     • Scienze e tecnologie geologiche 
     • Scienze e tecnologie per l'ambiente e il territorio 
     • Teoria e tecnologia della comunicazione 

Il corso in Teoria e tecnologia della comunicazione è tenuto in collaborazione con Psicologia. 

Studiare in Bicocca
L'accesso ai corsi triennali senza il numero programmato prevede il superamento di una prova che verte principalmente sulle conoscenze di matematica e logica. Per il primo anno, la maggioranza dei corsi di laurea triennale della Scuola di Scienze è a numero programmato. I corsi di laurea magistrale prevedono requisiti curricolari e competenze che sono specificati sul manifesto dei rispettivi corsi di laurea. 
Scienze offre inoltre dei precorsi di richiami di matematica e di metodologia dello studio universitario. Durante il primo anno, sono previsti corsi di recupero per chi non avesse superato il VPI. 

Contatti
Sito di Scienze: www.scienze.unimib.it 
Precorsi: http://www.scienze.unimib.it/?page_id=243
