Economia 
Presentazione
L'università offre 4 corsi di laurea triennale nell'ambito delle discipline economiche, l'iscrizione ai quali è preceduta da un test di valutazione che, se non superato, permette comunque l'iscrizione al corso di laurea ma non l'accesso agli esami. Sono presenti inoltre quattro corsi di laurea magistrale, due dei quali hanno dei requisiti minimi di voto/media ponderata per l'accesso, che avviene solo dopo un colloquio.
Corsi di laurea triennale
     • Economia delle banche, delle assicurazioni e degli intermediari finanziari. 
     • Economia e amministrazione delle imprese. 
     • Economia e commercio. 
     • Marketing, comunicazione aziendale e mercati globali. 
Corsi di laurea magistrale
     • Economia e Finanza. 
     • Scienze dell'economia. 
     • Scienze Economico-Aziendali. 
     • Marketing e Mercati Globali. 
Per i corsi di laurea magistrale in Scienze Economico-Aziendali e Marketing e Mercati Globali l'iscrizione è condizionata dal soddisfacimento di requisiti minimi. Per il primo è necessario un voto di laurea triennale pare a 91/110 e/o una media ponderata di 21/30, ciò permette l'accesso al colloquio in base al quale è stilata una graduatoria per l'accesso. Per il secondo corso invece è necessario un voto di laurea triennale minimo pari a 94/110 e/o una media ponderata del 24/30. Anche in questo caso il soddisfacimento di questi requisiti permette l'accesso ad colloquio ed una valutazione in ingresso. Nel caso non si possedessero i requisiti gli studenti che volessero comunque iscriversi a questi corsi di laurea dovranno sostenere un test composto di dieci domande, dovendo ottenere 6/10 di risposte corrette; superato il test sarà possibile l'accesso al colloquio ed alla graduatoria. 
Contatti
Informazioni più dettagliate sono disponibili sul sito: http://www.economia.unimib.it/ 
