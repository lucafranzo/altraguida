3.4 Tasse
Le tasse in Bicocca si pagano attraverso due rate; la prima delle due, uguale per tutti, ammontava fino all'anno scorso a 635€ (554€ per gli studenti al primo anno fuori corso): da quest'anno questa voce è calata di 40€ anche grazie al contributo dei nostri rappresentanti in consiglio di amministrazione: 595€ (514€ per gli studenti al primo anno fuori corso). La seconda delle due rate è a differenza della prima diversa per fascia di reddito e corso di studi. 
Per pagarle è necessario stampare il MAV dal proprio profilo delle segreterie online e recarsi presso un qualsiasi sportello bancario. La prima rata dev'essere pagata, ogni anno, entro fine settembre, per potersi iscrivere all'anno successivo. Per pagare la seconda rata è necessaria la dichiarazione ISEEU, effettuabile gratuitamente presso uno dei Caf convenzionati con l'università, entro metà dicembre. 
Il bollettino con la seconda rata è disponibile solitamente verso Aprile e dev'essere pagata entro metà maggio. La quota è calcolata automaticamente dall'università in base al valore dell'Iseuu che le è stato trasmesso, secondo questa regola: 
       • ISEEU<14.000 rata uguale alla rata minima del proprio corso di studi (54 euro per area B, 81 euro per area C, 0 per area A) 
       • 14.000<ISEEU<35.000 [(ISEEU - 14.000,00) x 2,8%] x Coefficiente area + importo minimo di seconda rata relativo all'area. 
       • ISEEU>35.000 {[(35.000,00 - 14.000,00) x 2,8%] + [(ISEEU - 35.000,00) x 4%]} x Coefficiente area + importo minimo di seconda rata relativo all'area. 
Se il valore calcolato è minore di 5 euro, la seconda rata non viene addebitata. Se invece il valore supera l'importo massimo (2.330 euro per area A, 2850 per Area B, 3110 per Area C) la rata corrisponde semplicemente alla rata massima. In Area A si trovano Giurisprudenza, Economia, Scienze Statistiche, Sociologia, Psicologia, In Area B ci sono Scienze MMFFNN, Scienze della Formazione, le lauree dell'area sanitaria, il corso di laurea in Teoria e Tecnologia della Comunicazione. In Area C i Corsi di Laurea in Medicina e Chirurgia. 
ATTENZIONE: sono previste delle more in caso di pagamento delle tasse o di consegna della dichiarazione ISEEU dopo le scadenze previste. Ci sono anche delle tasse aggiuntive per richiedere i duplicati del libretto o del badge in caso di smarrimento.

3.5 Esoneri
Oltre alle borse di studio, per studenti in condizioni economiche disagiate, esistono alcuni tipi di esoneri parziali o totali dal pagamento delle tasse.Quelli totali sono previsti per gli studenti che hanno diritto (anche se non dovessero riceverla) ad una borsa di studio o ad un prestito d'onore del C.I.Di.S., per i beneficiari di una borsa di studio dell'Università, del Ministero degli Affari Esteri, per gli studenti con una disabilità superiore al 66%.
Sono previsti esoneri parziali per diverse tipologie di studenti, ad esempio per gli studenti lavoratori, studenti dipendenti dell'università stessa, studenti con invalidità compresa tra il 45% e il 5%. Per i dettagli consultate sempre la Guida dello Studente o il sito di Ateneo alla pagina "tasse/esoneri" (http://www.unimib.it/go/1755630489).

3.6 Piani di studi
Il piano di studi è l'elenco degli esami che lo studente deve sostenere per conseguire la laurea, comprendendo sia quelli obbligatori per ogni corso di laurea sia quelli a scelta dello studente. Il piano di studi viene compilato tramite le Segreterie On Line seguendo la procedura indicata, in un periodo stabilito annualmente dall'Ateneo. Non è necessario compilarlo ogni volta, se non si ha intenzione di modificare quello dell'anno passato. 
Ogni Corso di Laurea ha un regolamento che stabilisce il numero di esami, la tipologia e il numero di crediti da acquisire; quindi per avere un piano di studi in regola controllate le norme del vostro corso. Per maggiori indicazioni su come si compili un piano di studi, consultate la pagina del sito d'Ateneo o della vostra Facoltà. I piani di studi che non dovessero essere approvati possono essere ripresentati l'anno successivo o modificati quando si consegna la domanda di laurea.
