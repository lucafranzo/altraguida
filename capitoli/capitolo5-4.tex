5.10 Le associazioni con cui collaboriamo
KOB (Kollettivo Omosessuale Bicocca)
Il Kollettivo Omosessuale Bicocca è un gruppo studentesco LGBT nato nel 2003 all'interno dell'Università di Milano - Bicocca, a scopo aggregativo e culturale. 
All'interno dell'Ateneo promuove la cultura delle differenze relative all'orientamento sessuale e all'identità di genere con iniziative realizzate sia autonomamente che insieme ad altre realtà universitarie, organizzando attività rivolte al gruppo stesso e/o a tutti gli studenti dell'Università, come conferenze, cineforum e mostre. Collabora anche con le altre associazioni LGBT presenti sul territorio per costruire progetti e percorsi dedicati all'ambiente universitario e talvolta ospitando iniziative precedentemente proposte in altri contesti ma che possano essere considerate d'interesse anche per gli studenti. 
Tutto questo è permesso dai contributi che, per legge, ogni ateneo deve destinare a iniziative studentesche, ai quali il KOB accede presentando ogni anno i propri progetti per il Bando Mille Lire, valutati poi da un'apposita commissione. Nel corso degli anni il KOB ha organizzato cineforum, conferenze, feste studentesche e incontri informali in Università e fuori; è stato presente ai Pride e alle altre manifestazioni politiche e culturali LGBT; ha contribuito all'informazione e ai confronti in rete tramite siti web, social network, forum, etc. Ha partecipato attivamente alla ricostituzione del Coordinamento Arcobaleno, che riunisce le associazioni LGBT milanesi, ed ha preso parte all'iniziativa itinerante L'Amore Spiazza e all'organizzazione dei Pride locali e nazionali Inoltre si pone come occasione di incontro e confronto per le persone LGBT che vivono l'ambiente universitario o che interagiscono con questo anche solo occasionalmente. Fornisce un servizio di accoglienza per chi lo desidera, rivolto a studenti, professori e lavoratori dell'Ateneo, tramite il quale vengono valutate le esigenze individuali di chi usufruisce di tale servizio: dal desiderio di conoscere persone omosessuali magari per la prima volta, a quello di impegnarsi in attività organizzative, oppure per essere indirizzati ad altre realtà LGBT che si occupano in maniera più specifica di determinati temi, o ancora per essere indirizzati verso servizi di aiuto e sostegno in casi più particolari. 
Il gruppo è composto da circa 250 iscritti con un turn over di circa 3 anni. Una ventina di persone si occupano attivamente della gestione e dell'organizzazione del gruppo e delle sue attività. Essendo un collettivo e non un'associazione con statuto, il KOB non ha una gerarchia di riferimento, bensì referenti per ogni specifica iniziativa. 

ASB (Associazione Studenti Bicocca)
L'Associazione Studenti Bicocca (ASB) è un'associazione culturale, senza fini di lucro, nata nel 2002 dalla collaborazione di studenti provenienti da diverse facoltà dell'Università degli Studi di Milano Bicocca. ASB è quindi un punto di incontro di studenti di diverse facoltà, dove si ha modo sia di socializzare che promuovere concretamente i propri interessi, come conferenze, feste, concorsi fotografici, aperitivi culturali, giornate di sensibilizzazione sulla sostenibilità. ASB cura altresì il forum degli studenti della Bicocca (www.studentibicocca.org), con più di 38.000 iscritti e 5.000 accessi giornalieri, dove i ragazzi hanno modo di confrontarsi sui più disparati temi nonché sulle materie di esame, trovando anche materiali gratuiti da scaricare, condivisi da altri studenti. Per informazioni scrivete ad asb@studentibicocca.org o visitate www.studentibicocca.org.
